\section{Reference frames}
\section{Dinghy boat: The Laser class}

%% Include equations here 
\begin{equation}
\label{eq_vel}
\begin{aligned}
V_{aw} & = V_{tw,b} - V_{c,b} - V_{boat}
\end{aligned}
\end {equation}



$V_{aw}$ and its direction $( \alpha_{aw})$ are highly important because it is the velocity perceived by the moving sailboat and its is given by the difference between the wind and the boat; if the boat is moving in the same direction as the wind then the apparent velocity increase. \par 
%% If height exceed width, then the effective sail area is A=pi h/4. The area of the air used in our calculations depends on the height of the sail, this is explained in the prandt and . The force of the boat is given by  Fboat= Fwind cos (a sail-app wind) cos (avw- sail-app wind). If the boat is in downwind condition veq=wind vel/( 1+ (beta)^0.5 ) and beta = drag factor / (density air * Area sail)

The wind, current and boat velocities interaction is represented with the velocity triangle.  Figure \ref{vel_triangle}, shows the relation between wind and boat (yacht) velocities and the additional resulting velocities; the apparent wind velocity ($V_{aw}$) and speed made good,  also know as velocity made good (vector notation). The figure also shows how the true and apparent wind angle are made. It is important to explain that the velocity of the current is considered in the boat's velocity, equation \ref{eq_vel}; where \textit{tw} is the true wind velocity, \textit{b} means respect to the boat and \textit{c} is the current velocity, $\alpha$ is the sail angle and \textit{V} is the velocity vector. \par